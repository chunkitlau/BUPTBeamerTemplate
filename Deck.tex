% Copyright 2019 Clara Eleonore Pavillet

% Author: Clara Eleonore Pavillet (original author), 
%         Chunkit Lau (modified for BUPT)
% Description: This is an unofficial Beijing University of Posts and Telecommunications(BUPT) Beamer Template I made from scratch. Feel free to use it, modify it, share it.
% Version: 0.1.0

\documentclass{beamer}
\input{Theme/Packages.tex}
\usetheme{BUPT}

\title{Mathematical and Computational Modelling}
\titlegraphic{\includegraphics[width=2cm]{Theme/Logos/BUPT_badge.png}}
\author{Chunkit Lau}
\institute{Beijing University of Posts and Telecommunications}
\date{} %\today

\begin{document}

{\setbeamertemplate{footline}{} 
\frame{\titlepage}}

\section*{Outline}\begin{frame}{Outline}\tableofcontents\end{frame}

\section{Text}
    \begin{frame}[plain]
        \vfill
      \centering
      \begin{beamercolorbox}[sep=8pt,center,shadow=true,rounded=true]{title}
        \usebeamerfont{title}\insertsectionhead\par%
        \color{BUPTblue}\noindent\rule{10cm}{1pt} \\
        \LARGE{\faFileTextO}
      \end{beamercolorbox}
      \vfill
  \end{frame}


\subsection{Welcome}
\begin{frame}{Welcome}
This BUPT-themed template is derived from the OxPav template made by Clara Eleonore Pavillet \href{https://github.com/CEPav}{\faGithub} for Oxford University.

\begin{enumerate}
    \item Simple
    \item Clean
    \item BUPT University Colours
\end{enumerate}
\vspace{1cm}
\begin{center}
    Enjoy! \faSmileO
\end{center}
\end{frame}


\section{Equations}
    \begin{frame}[plain]
        \vfill
      \centering
      \begin{beamercolorbox}[sep=8pt,center,shadow=true,rounded=true]{title}
        \usebeamerfont{title}\insertsectionhead\par%
        \color{BUPTblue}\noindent\rule{10cm}{1pt} \\
        \LARGE{\faFileTextO}
      \end{beamercolorbox}
      \vfill
  \end{frame}
  
\subsection{Example}
\begin{frame}{Example}
\only<1>{
Let \(p(x)=\mathcal{N}(\mu\textsubscript{1},\,\sigma^{2}\textsubscript{1})\) and \(q(x)=\mathcal{N}(\mu\textsubscript{2},\,\sigma^{2}\textsubscript{2})\): \\
\begin{equation}
\mathcal{N}=\frac{1}{\sigma\,\sqrt{2\,\pi}}\,\E^{-\frac{\left(x-\mu\right)^2}{2\,\sigma^2} }
\end{equation}}
\only<2>{
Kullback-Leibler divergence for continuous probabilities:
\begin{align*}
	D(p,q)=&\int p(x) \log \frac{p(x)}{q(x)}\ud x\\
    =& \int p(x) \,\ln p(x) \ud x -\int p(x) \,\ln q(x) \ud x\\
	=&\,\frac{1}{2} \ln\left(2\,\pi\,\sigma_2^{2}\right) +\frac{\sigma_1^{2}+\left(\mu_1-\mu_2 \right)^2 }{2\,\sigma_2^2}-\frac{1}{2}\left( 1+\ln 2\,\pi\,\sigma_1^2\right) \\
	=&\,\ln\frac{\sigma_2}{\sigma_1} +\frac{\sigma_1^{2}+\left(\mu_1-\mu_2 \right)^2 }{2\,\sigma_2^2}-\frac{1}{2}
\end{align*}
}
\end{frame}

\section{Code}
    \begin{frame}[plain]
        \vfill
      \centering
      \begin{beamercolorbox}[sep=8pt,center,shadow=true,rounded=true]{title}
        \usebeamerfont{title}\insertsectionhead\par%
        \color{BUPTblue}\noindent\rule{10cm}{1pt} \\
        \LARGE{\faFileCodeO}
      \end{beamercolorbox}
      \vfill
  \end{frame}
  
\subsection{Example}
\begin{frame}[fragile]{Example}
\begin{block}{Greatest Common Divisor}
\begin{lstlisting}[firstnumber=1, label=glabels, xleftmargin=10pt] 
def greatest_c_remainder(a,b):
	'''Greatest common divisor of a and b'''
	r = a % b
	if r == 0:
		return b
	else:
		m = b
		n = r
	return greatest_c_remainder(m,n)

\end{lstlisting}
\end{block}
\end{frame}

\end{document}

